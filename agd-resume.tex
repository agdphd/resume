\documentclass[11pt,colorlinks,urlcolor=blue]{moderncv}

\moderncvstyle{casual}
\moderncvcolor{black}

\usepackage[margin=.8in,letterpaper]{geometry}

\usepackage[T1]{fontenc}
\usepackage[utf8]{inputenc}
\usepackage{libertine}
\usepackage[libertine]{newtxmath}
\usepackage{inconsolata}

\usepackage{xpatch}
\xpatchcmd{\cventry}{\small}{}{}{}

\nopagenumbers
\AtEndPreamble{%
  \AtBeginDocument{%
    \ifthenelse{\pageref{lastpage}>1}{
      \newlength{\pagenumberwidth}%
      \settowidth{\pagenumberwidth}{\color{color2}\addressfont\itshape\thepage/\pageref*{lastpage}}%
      \fancypagestyle{plain}{%
        \fancyfoot[r]{\parbox[b]{\pagenumberwidth}{\color{color2}\pagenumberfont\thepage/\pageref*{lastpage}}}}
      \pagestyle{plain}
      \AtEndDocument{\label{lastpage}}
    }{}
  }
}

\let\oldnamefont\namefont
\renewcommand*{\namefont}{\oldnamefont\sffamily}

\let\oldtitlefont\titlefont
\renewcommand*{\titlefont}{\oldtitlefont\sffamily}

\let\oldsectionfont\sectionfont
\renewcommand*{\sectionfont}{\oldsectionfont\bfseries\sffamily}

\let\oldsubsectionfont\subsectionfont
\renewcommand*{\subsectionfont}{\oldsubsectionfont\sffamily}

\usepackage{csquotes}

\usepackage{microtype}

\name{Andrew R.}{Gainer-Dewar, Ph.D.}
\email{andrew.gainer.dewar@gmail.com}
\social[github]{agdphd}
\social[linkedin]{agdphd}

\begin{document}
\vspace*{-.5in}
\maketitle
\vspace*{-.5in}

\setlength{\parskip}{1ex}

\section{Summary of qualifications}
I'm a software engineer and mathematician with development experience in C++, Java, and Python.

I love building elegant, useful solutions to challenging problems.
I'm passionate about writing code that's stable, efficient, and well-tested, and I'm eager to build experience doing so in production and at scale.

My recent work includes applying computational geometry to RNA folding prediction, using combinatorics to analyze cell signalling networks, and implementing graph algorithms for the open-source library JGraphT.

\section{Languages and skills}
\cvitem{\textbf{Python} (7yr)}{
  I use Python extensively for exploratory data analysis and workflow management.
  I used the Python-based computer algebra system SageMath for my thesis, and ultimately contributed bug fixes and implementations of several algebraic objects to the project.
  Python is my language of choice for hobby competitive programming.
}

\cvitem{\textbf{Java} (1yr)}{
  I rewrote a research application from the ground up to use the new Cytoscape 3.0 network-analysis platform and improve performance.
  These ten thousand lines of code run the gamut from UIs in Swing to a class hierarchy implementing user-selectable alternatives for combinatorial algorithms and include extensive unit tests.
  I also contributed several algorithms and bug fixes to the open-source JGraphT library.
}

\cvitem{\textbf{C++} (2yr)}{
  I implemented a variety of algorithms for the Minimal Hitting Set Enumeration problem, which I now offer as a reusable library under an open license on GitHub.
  I refactored the \texttt{gtmfe} RNA folding prediction software to use rational arithmetic from the \texttt{GMP} library, then used it as a module for a computational geometry package I wrote using the \texttt{CGAL} library.
}

\cvitem{\textbf{Math} (10yr+)}{
  My Ph.D.~thesis was in enumerative combinatorics and graph theory.
  I've also worked on computational geometry and combinatorial biology, and I'm comfortable picking up new domains as needed.
  My mathematical training has taught me to solve a problem by finding the right abstractions to break it down into small bits, then using the really big hammers in the literature (and maybe a few of their own invention) to smash them.
  That's how I like to code, too.
}

\cvitem{\bfseries Miscellany}{
  I have used Linux (primarily Debian) in personal and work environments for nearly twenty years and am completely comfortable on the command line and with light systems administration.
  I built an algorithm benchmarking system using Docker and have deployed computations on clusters.
  I use Git for every line of code I produce, including my academic writing, my dotfiles, and this r\'{e}sum\'{e}; many of these projects are available under open licenses on GitHub, and I have contributed to several community projects there as well.
}

\section{Professional experience}
\cventry{2015--07/2016}{University postdoctoral fellow}{UConn Health}{Farmington, CT}{}{
  \begin{itemize}
  \item
    Rewrote \texttt{OCSANA}, a Java/\texttt{Cytoscape} application applying network-theoretic and combinatorial algorithms to systems biology, achieving $10000\times$ speedup on typical inputs
  \item
    Implemented custom benchmark framework with Python and Docker to evaluate two dozen algorithms for core combinatorial problem
  \item
    Implemented several of the above algorithms in C++ as an open-source library
  \item
    Contributed to the open-source JGraphT library
  \end{itemize}
}

\cventry{2014--2015}{Visiting assistant professor of mathematics}{Hobart and William Smith Colleges}{Geneva, NY}{}{
  \begin{itemize}
  \item
    Wrote \texttt{pmfe}, a C++ application integrating RNA folding prediction algorithms with computational geometry and geometric combinatorics to study parameter sensitivity
  \item
    Full-time teaching
  \end{itemize}
}

\cventry{2012--2014}{Visiting assistant professor of mathematics}{Carleton College}{Northfield, MN}{}{
  \begin{itemize}
  \item
    Full-time teaching
  \end{itemize}
}

\cventry{2007--2012}{Graduate fellow}{Brandeis University}{Waltham, MA}{}{
  \begin{itemize}
  \item
    Completed thesis in enumerative combinatorics and graph theory
  \item
    Contributed to open-source SageMath computer algebra system
  \end{itemize}
}

\section{Education}
\cventry{2007--2012}{Ph.D.\ (Mathematics)}{Brandeis University}{Waltham, MA}{}{}

\cventry{2003--2007}{B.S.\ (Mathematics) \& B.A.\ (Philosophy)}{Mercer University}{Macon, GA}{}{}
\end{document}

%%% Local Variables:
%%% mode: latex
%%% TeX-master: t
%%% End:
