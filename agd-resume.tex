\documentclass[12pt,colorlinks,urlcolor=blue]{moderncv}

\moderncvstyle{casual}
\moderncvcolor{grey}

\usepackage[margin=0.7in,letterpaper]{geometry}

\usepackage[T1]{fontenc}
\usepackage[utf8]{inputenc}
\usepackage{libertine}
\usepackage[libertine]{newtxmath}
\usepackage{inconsolata}

\usepackage{comment}

\let\oldnamefont\namefont
\renewcommand*{\namefont}{\oldnamefont\sffamily}

\let\oldtitlefont\titlefont
\renewcommand*{\titlefont}{\oldtitlefont\sffamily}

\let\oldsectionfont\sectionfont
\renewcommand*{\sectionfont}{\oldsectionfont\bfseries\sffamily}

\let\oldsubsectionfont\subsectionfont
\renewcommand*{\subsectionfont}{\oldsubsectionfont\sffamily}

\usepackage{csquotes}

\usepackage{microtype}

\name{Andrew R.}{Gainer-Dewar, Ph.D.}
\email{andrew.gainer.dewar@gmail.com}
\social[github]{agdphd}
\social[linkedin]{agdphd}

\begin{document}
\vspace*{-.5in}
\maketitle
\vspace*{-.5in}

\section{Summary of qualifications}
I'm a software engineer and mathematician with development experience in C++, Java, and Python.

I love building elegant, useful solutions to challenging problems.
I'm passionate about writing code that's stable, efficient, and well-tested, and I'm eager to build experience doing so in production and at scale.

My recent work includes applying computational geometry to RNA folding prediction, using combinatorics to analyze cell signalling networks, and implementing graph algorithms for the open-source library JGraphT.

\section{Languages and skills}
\cvitem{\textbf{Python} (7yr)}{
  I use Python extensively for scripting data analysis and workflow management.
  I also used the Python-based computer algebra system SageMath for my thesis, including implementing several classes and submitting a variety of bugfixes.
  Python is also my language of choice for hobby competitive programming.
  Its extremely lightweight and flexible style is awesome, but I can't help but miss static types when they're gone.
}

\cvitem{\textbf{Java} (1yr)}{
  I rewrote an application that runs on the Cytoscape network-analysis platform from the ground up.
  These ten thousand lines of code run the gamut from UI implementation in Swing to a class heirarchy implementing user-selectable alternatives for combinatorial algorithms.
  I also submitted several algorithms and bugfixes to the open-source JGraphT library.
  I greatly appreciate the new stream functional features in Java 8 and the comfort of static types, even if I do sometimes have to write a \texttt{RunnerTaskContextBuilderFactory}.
}

\cvitem{\textbf{C++} (2yr)}{
  I rebuilt the \texttt{gtmfe} RNA folding prediction software to use rational arithmetic from the \texttt{GMP} library, then used it as a module for a computational geometry package I wrote using the \texttt{CGAL} library.
  I also implemented a variety of algorithms for the Minimal Hitting Set Enumeration problem as library modules which are available under an open license on GitHub.
  C++ is currently my weakest language of these three, but I'm happy to pick it up again.
}

\cvitem{\textbf{Math} (10yr+)}{
  My Ph.D.~thesis was in enumerative combinatorics and graph theory.
  I've also worked on computational geometry and combinatorial biology, and I'm comfortable picking up new domains and skills as needed.
  Mathematicians solve a problem by breaking it down into bite-size pieces, finding the right abstractions, and then using the really big hammers in the literature (and maybe a few of their own invention) to smash it.
  That's how I like to code, too.
}

\cvitem{\bfseries Miscellany}{
  I have used Linux (primarily Debian) in personal and work environments for nearly twenty years and am completely comfortable on the command line and with lightweight systems administration.
  I built an algorithm benchmarking system using Docker and have deployed computations on clusters.
  I use Git for nearly every line of code I produce (including my Ph.D.~thesis, my dotfiles, and this r\'{e}sum\'{e}) and have numerous projects on GitHub.
}

\clearpage

\section{Professional experience}
\cventry{2015--07/2016}{University postdoctoral fellow}{UConn Health}{Farmington, CT}{}{
  \begin{itemize}
  \item
    Rewrote \texttt{OCSANA}, a Java/\texttt{Cytoscape} application applying network-theoretic and combinatorial algorithms to problems in systems biology, achieving $10000\times$ speedup on typical inputs
  \item
    Implemented custom benchmark framework with Python and Docker to evaluate two dozen algorithms for core combinatorial problem
  \item
    Implemented several of the above algorithms in C++ and released them as an open-source library
  \item
    Submitted several algorithms and bugfixes to the open-source JGraphT library
  \end{itemize}
}

\cventry{2014--2015}{Visiting assistant professor of mathematics}{Hobart and William Smith Colleges}{Geneva, NY}{}{
  \begin{itemize}
  \item
    Wrote \texttt{pmfe}, a C++ application integrating state-of-the-art RNA folding prediction algorithms with computational geometry and geometric combinatorics to study parameter sensitivity
  \item
    Full-time teaching
  \end{itemize}
}

\cventry{2012--2014}{Visiting assistant professor of mathematics}{Carleton College}{Northfield, MN}{}{
  \begin{itemize}
  \item
    Full-time teaching
  \end{itemize}
}

\cventry{2007--2012}{Graduate fellow}{Brandeis University}{Waltham, MA}{}{
  \begin{itemize}
  \item
    Completed thesis in enumerative combinatorics and graph theory
  \item
    Submitted several classes and bugfixes for Python-based SageMath computer algebra system
  \end{itemize}
}

\section{Education}
\cventry{2007--2012}{Ph.D.\ (Mathematics)}{Brandeis University}{Waltham, MA}{}{}

\cventry{2003--2007}{B.S.\ (Mathematics) \& B.A.\ (Philosophy)}{Mercer University}{Macon, GA}{}{}
\end{document}

%%% Local Variables:
%%% mode: latex
%%% TeX-master: t
%%% End:
